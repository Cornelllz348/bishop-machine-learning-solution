\pb{1.2}

We use the same notation as in the previous problem and still rewrite
the loss function in matrix form as follows: 
\[
\widetilde{E}(w)=\frac{1}{2}\left\langle Xw-t,Xw-t\right\rangle +\frac{\lambda}{2}\left\langle w,w\right\rangle .
\]
Still we differentiate the expression. Note that if we let $\varphi(w)=\frac{\lambda}{2}\left\langle w,w\right\rangle ,$we
have that 
\begin{align*}
\varphi(w+h) & =\frac{\lambda}{2}\left(w+h\right)^{T}\left(w+h\right)\\
 & =\frac{\lambda}{2}\left(w^{T}w+w^{T}h+h^{T}x+\left\Vert h\right\Vert \right)\\
 & =\varphi\left(w\right)+\left\langle \lambda w,h\right\rangle +\underbrace{\frac{\lambda}{2}\left\Vert h\right\Vert }_{=o(\left\Vert h\right\Vert )}.
\end{align*}
Therefore, $\nabla\varphi(w)=\lambda w,$ and as a result 
\[
\nabla\widetilde{E}(w)=\nabla E(w)+\nabla\varphi(w)=X^{T}(Xw-t)+\lambda w.
\]
Setting it to zero: 
\[
X^{T}(Xw-t)+\lambda w=0\iff(X^{T}X+\lambda I)w=X^{T}t.
\]
Hence, $(X^{T}X+\lambda I)$ and $X^{T}t$ are the corresponding matrices. 



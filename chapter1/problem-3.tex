\pb{1.3}

According to the Bayes formula, we get that 
\begin{align*}
P\left(\text{apple}\right) & =P\left(\text{apple}\vert\text{r}\right)P\left(\text{r}\right)+P\left(\text{apple}\vert\text{g}\right)P\left(\text{g}\right)+P\left(\text{apple}\vert\text{b}\right)P\left(\text{b}\right)\\
 & =\frac{3}{10}\cdot\frac{2}{10}+\frac{1}{2}\frac{2}{10}+\frac{3}{10}\frac{6}{10}=\frac{17}{50}.
\end{align*}
And again, we can use formula to get 
\begin{align*}
P\left(\text{g}\vert\text{orange}\right) & =\frac{P\left(\text{orange}\vert\text{g}\right)P\left(\text{g}\right)}{P\left(\text{orange}\vert\text{g}\right)P\left(\text{g}\right)+P\left(\text{orange}\vert\text{b}\right)P\left(\text{b}\right)+P\left(\text{orange}\vert\text{r}\right)P\left(\text{r}\right)}\\
 & =\frac{\frac{3}{10}\frac{6}{10}}{\frac{3}{10}\frac{6}{10}+\frac{2}{10}\frac{1}{2}+\frac{2}{10}\frac{4}{10}}\\
 & =\frac{1}{2}.
\end{align*}
 
